\chapter{Statement of Ethics}

This chapter examines the legal, social, ethical and professional (LSEP) considerations relating to the project.

\section{Legal Considerations}

Throughout the development and implementation of this project, careful attention has been paid to ensuring full compliance with relevant legal frameworks. 
The Computer Misuse Act (1990) has been strictly adhered to in all aspects of the work, the software developed contains no malicious code that could lead to security breaches or unauthorised system access. 
With regard to intellectual property considerations, the project exclusively utilises open-source libraries, frameworks and assets that are distributed under licenses permitting their use in this context. 
All third-party components have been legally obtained through official channels and implemented in accordance with their respective licensing terms. 
Throughout the entirety of the project lifecycle, no user data has been collected, processed, or stored in any capacity. All data used in training any models was created by me.
This eliminates concerns related to data protection legislation such as the General Data Protection Regulation (2018) and the Data Protection Act (1998), 
thereby circumventing potential privacy issues that might otherwise arise from the management of personal information. 
The absence of data collection mechanisms in the system architecture fundamentally removes associated legal obligations regarding data security, retention periods, 
and subject access requests that would normally need to be addressed in projects involving user data processing.
This approach not only ensures legal compliance but also aligns with the principles of open-source software development and academic integrity that underpin this work.

\section{Social Considerations}

This project has been developed with careful consideration of its social implications and potential impact. 
The reinforcement learning application created through this work has been designed specifically as an educational and demonstrative tool, rather than for deployment in scenarios where it could affect individuals or communities (for example as a cheating tool). 
By focusing on creating a harmless demonstration of reinforcement learning principles in action, the project prioritises transparency and accessibility in artificial intelligence education. 
The implementation deliberately avoids mechanisms that could introduce bias or unfair treatment, thereby sidestepping many of the social concerns that typically accompany machine learning systems deployed in consequential domains. 
Furthermore, the project's educational nature contributes positively to the democratisation of knowledge about advanced computational techniques, potentially inspiring further academic inquiry and innovation in this rapidly evolving field. 
This approach reflects a conscientious effort to advance technical capabilities while maintaining awareness of the broader social context in which technology operates.

\section{Ethical Considerations}

The development of this project has been guided by fundamental ethical principles and conducted in strict accordance with the University of Surrey's Ethics Guide and Ethics for Teaching and Research Policy. 
Throughout all stages of research and implementation, meticulous attention was paid to maintaining ethical standards, including research integrity, honesty in reporting results, and transparency in methodologies. 
The project involved no human participants or personal data collection, thus minimizing potential ethical concerns regarding privacy, consent, or potential harm.
All experimental results presented in this dissertation accurately reflect the actual performance of the developed systems, with no selective reporting or manipulation of outcomes. 
Additionally, the project maintains complete transparency about its capabilities and limitations, avoiding any exaggeration of the system's effectiveness or potential applications. 
No conflicts of interest arose during any phase of this work, and all findings have been reported with full integrity.

\section{Professional Considerations}

Professional standards and responsibilities have guided this project throughout its lifecycle. 
The development process adhered to established frameworks including the British Computer Society (BCS) Code of Conduct and the ACM Code of Ethics' Professional Responsibilities guidelines.
As this research was conducted at the University of Surrey, strict compliance with institutional policies was maintained at all stages. 
This included following the university's Ethics Guide, Ethics for Teaching and Research Policy, and Code of Practice for Patenting and Exploitation of Inventions.
The work demonstrates commitment to professional integrity through transparent documentation, proper attribution of sources, and responsible development practices.
