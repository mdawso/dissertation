\chapter{Design Decisions}

Over the course of the development of my project, I had to make numerous design decisions for each part. 
This chapter lays out and explains the design decisions I made, along with justifications for them.

\section{Version Control System}

Throughout the development of the project, I will use the industry-standard Git version control system. 
This allows me to keep snapshots of the project over time, so I can revert if anything goes wrong, as well as keeping the project backed up to GitHub.

\section{Development of the Game}

To develop the game frontend, I needed to choose a game engine or game development framework. Throughout my research, I considered three different options, Godot, Unity and Pygame.

I chose the Godot game engine for the following reasons:
\begin{itemize}
    \item Godot is open-source and free to use, which aligns with academic project requirements.
    \item It is natively compatible with Git, as all the resources are plain text based.
    \item It is extremely lightweight to develop with, not requiring an install and being 130MB in size.
    \item It has a simple and intuitive user interface that I already have experience using, allowing for faster development.
    \item The engine uses GDScript, a Python-like language that is easy to learn and use.
    \item Godot has robust 2D capabilities, which matched the requirements of my game project.
    \item It has a supportive community and extensive documentation, helpful for troubleshooting.
    \item The engine's node-based architecture provides flexibility in designing game components.
    \item It offers cross-platform deployment options, making the game accessible on multiple devices. This is important because the university machines run Linux, while my home machine runs Windows, and I need to ensure it runs on both.
    \item The engine features networking features, which will be essential in connecting to the model as Godot does not natively support any machine learning libraries.
\end{itemize}

\section{Development of the Model}

To develop the model backend, I chose to use PyTorch as my machine learning framework. PyTorch offers several advantages that made it suitable for this project:

\begin{itemize}
    \item PyTorch has a flexible and intuitive API, making it easier to develop and debug models.
    \item It has extensive community support and documentation, providing solutions to common problems.
    \item The framework includes built-in tools for neural network development and training.
    \item It supports GPU acceleration, which significantly speeds up the training process.
    \item PyTorch's dynamic computation graph allows for more flexible model designs.
    \item It integrates well with Python data science libraries like NumPy and Pandas.
    \item The framework has robust support for reinforcement learning algorithms, which was essential for my game AI.
    \item It allows for easy deployment of models through TorchServe or simple API integration.
\end{itemize}