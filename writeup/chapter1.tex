\chapter{Introduction}

In the area of video games, Artificial Intelligence (AI) has been used for many years to create non-player characters (NPCs) that can interact with players in a believable way.
This is often done using finite state machines (FSMs) or behaviour trees, which allow NPCs to react to player actions in a way that seems intelligent. 
However, these methods are by definition deterministic, meaning that the same input will always produce the same output, which can lead to predictable and uninteresting gameplay.
Additionally, these methods require a lot of manual tuning and design work to create interesting behaviours, which can be time-consuming and expensive, epecially for large or procedurally generated worlds.
In recent years, there has been a growing interest in using machine learning (ML) techniques to create more dynamic and adaptive NPCs.
One of the most promising approaches to this is deep reinforcement learning (DRL), which combines deep learning with reinforcement learning (RL) to create agents that can learn to play games by interacting with their environment.
This approach has been used successfully in a number of different games, including Atari games and board games like Go.

\section{Background}

Background text

\section{Aims and Objectives}

Aims and objectives text

